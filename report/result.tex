\section{Results and Interpretation}


\subsection{Performance}

We searched in the source code of Android framework since version 4.0 for string ?/dev/random? and found no reference to it. We also captured all read operation on /dev/random in a daily use of a Nexus 7 tablet since boot, and found only once, process \verb|wpa_supplicant| read from it. It was a WPA authenticator and it used /dev/random to initialize its own PRNG.

All Android applications who want high quality random number are recommended to use a Java class named SecureRandom, which on initialization read from /dev/urandom to seed its generator.

These results indicate that the performance issue cannot be attributed to blocking read of /dev/random since there are nearly none.

There may be some links between the entropy collecting routine and the poor performance. We noticed that since Linux kernel 3.7, the input event entropy source handler in random.c source code has been restructured (CITE: commit 4369c64c) to improve the performance on multitouch devices by batching the input events. Some developers commenting on Issue 42265 also suggested that the performance issue is due to suboptimal order of event process and entropy accumulation. We cannot tell if there would be any improvement since we did not successfully install a kernel later than 3.7 on our devices, but we do suggest users keep their software up-to-date since the changes are quite reasonable and promising.

\subsection{Security}

\subsubsection{Entropy Sources}

[TABLE]

Figure 1 suggests that the Android devices have sufficient source of entropy. The entropy contributed merely by disk events is enough to have /dev/urandom properly seeded before any user applications can run.


\subsubsection{User Applications}

Figure 2 and Table 2 show that the time when the user could start the first application is after the time when the amount of entropy first crossed the 192 threshold, which means the first read from /dev/urandom will cause the non-blocking pool extract at least 10 bytes (\verb|EXTRACT_SIZE| defined in random.c). The 80 bits entropy is enough to prevent any attacker trying to predict the random number. There will be more entropy mixed in when user touches or slides on the screen as shown in table 3. If it is the first boot, users will be asked to setup the wireless network and enter Google account information, which cause a lot of input events and further secure the /dev/urandom device. 

Therefore we claim that the user applications are able to enjoy high quality random numbers from the beginning.

\subsubsection{Saving and Restoring the Random Seed}

Android framework has a system service named EntropyMixer which replaces the traditional init script to restore the random seed at boot. There are two major differences between them that may cause security issue:

\begin{itemize}

\item EntropyMixer is an Android service which starts after the Android framework is initialized. Any processes starting before are not protected by the saved entropy. /dev/urandom is merely seeded by the time of boot and machine information before that.

\item EntropyMixer reads /dev/urandom to save the random seed on boot and then saves every 3 hours instead of before rebooting or halting. This may be a better choice since cell phones are not likely to reboot or halt normally, but it also means the amount of entropy of the saved seed is less than or equal to the amount of entropy of the time of first boot, if the device is rebooted within three hours of its first boot. 

\end{itemize}

\subsubsection{Shared Stack Canary}

[Fig: Zygore forks all]

We found that virtually all processes need to read from /dev/urandom at their start. We looked into why and found that recent Android versions since 4.0 adopted the stack protector mechanism, as known as stack canaries to protect the Android framework processes as well as native code bridged by Java Native Interface in user applications.

The stack protector works with the help of two components:
\begin{itemize}

\item Android NDK compiler (gcc): insert putting and checking code in function prologues and epilogues. The canary value is read from a global variable.
\item The standard C library (bionic): initialize the canary value by reading 4 bytes from /dev/urandom and put it in a global variable. The initialization routine is a library constructor which is invoked by the dynamic linker after mapping the shared library into the process? memory space, typically when the process is exec()ed.
\end{itemize}

However, the way in which the canary value is initialized could be problematic on Android platform. All system services and user applications running on the Dalvik VM are forked from a process called Zygote. By leveraging the Copy-on-Write fork on Linux, Android apps do not need to load the Dalvik VM execution image or the shared library repeatedly on start, leading to better performance. But it also means that there is no chance for an app to invoke the library constructor which setup the canary value. Therefore all Android processes share one stack canary.

We made an app (CITE: our repository) to show its canary value on the screen and compared this value to what \verb|app_process| (which then forked Zygote) read from /dev/urandom on its start. The result verified this vulnerability. 

\subsubsection{Predictable Stack Canary}

Combining section 2.2 and 2.3, we found that it is possible to predict the canary value shared by all Android apps and Zygote. Note that EntropyMixer is also a system service running on Dalvik Vm and thus is going to be forked from Zygote. Therefore, when Zygote sets up its canary value, the /dev/urandom is only initialized by its internal routine (\verb|std_initialise|)  with the system information and boot time.

Due to multiprocessing and scheduling, the canary value is not a constant when the initial state of /dev/urandom pool is fixed.  We can model the canary value as a function of the boot time (the initial state) and the number of extractions from /dev/urandom that happened before Zygote read from /dev/urandom. In order to measure the entropy brought by multiprocessing and scheduling, we collected the number of bytes extracted before Zygote read from 2xx cold boots.

[Figure: histogram] 

If the attacker has prior knowledge of the distribution of the number of extractions before, he could guess the ranges in a higher probability first order, which gives an expected number of guess of 121, equivalent to 7 bits of entropy.

The boot time is acquired by getnstimeofday() function which supports resolution of nanosecond, but we found only *** precision on our devices. 

Hence there would be ??? bytes for the expected case and at most ??? bytes for the worst case.

Although the computation needed to predict the value of canary may be comparable to simply guessing the 32-bit canary, this vulnerability nullifies the additional security brought by extending the length of the canary.

\subsubsection{Nullified ASLR}

Since 4.0, Android has turned on address space layout randomization with the support of the kernel. Despite of only 8 bits of entropy is added for each mapped region, due to the fork nature of Zygote, all Android apps share one address layout, including the same base addresses of the stack, the heap and the standard C library.

