\section{Conclusion}

We have investigated the low entropy issue on Android and shown that the poor performance is not caused by blocking reads to /dev/random, since there are none of them. Starting from this issue, we also have investigated possible security vulnerabilities and we have shown that the PRNGs on Android devices are provide secure pseudorandom numbers for all user applications that requires them, because the entropy generated during boot is enough to seed the PRNG properly. However we have found that the stack protecting mechanism initializes the canary value shared by all future Android applications by a value read from the PRNG at an early time when the PRNG is not ready. We have shown that this vulnerability could result in less computation for guessing the canary value, depending on the knowledge of the attacker on the boot time. Furthermore, we have also shown that the shared values vulnerability could leak not only the canary value but the base addresses information, effectively rendering the address space layout randomization mechanism on Android. We have discussed possible attacks against this vulnerability and implemented on of them. In the end, we have proposed a fix to the shared canary vulnerability and argued that the general shared values vulnerability is difficult to fix.
We look into the issue that the amount of entropy kept by the pseudorandom number generator (PRNG) of Android is constantly low. We find that the accusation against this issue of causing poor performance and low frame rate experienced by users is ungrounded. We also investigate possible security vulnerabilities resulting from this issue. We find that this issue does not affect the quality of random numbers that are generated by the PRNG and used in Android applications because recent Android devices do not lack entropy sources. However, we identify a vulnerability in which the stack canary for all future Android applications is generated earlier than the PRNG is properly setup. This vulnerbility makes stack overflow simpler and threats Android applications linked with native code (through NDK) as well as Dalvik VM instances. An attacker could nullify the stack protecting mechanism, given the knowledge of the time of boot or a malicious app running on the victim device. This vulnerability also affects the address space layout randomization (ALSR) mechanism on Android, and can turn it from a weak protection to void. We discuss in this paper several possiable attacks against this vulnerbility as well as ways of defending. As this vulnerbility is rooted in an essential Android design choice since the very first beginning, it is difficult to fix.

