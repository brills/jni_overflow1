\section{Related Work}


Previous work was done to analysis the security of the Linux PRNG, especially the non-blocking pool, (CITE: linux PRNG revisited) from a cryptographic point of view. It showed that given the internal state not known to the attacker, predicting the output of the non-blocking pool is as hard as reversing the SHA-1 hash function, even if there is not any entropy mixed in. We base our work on this assumption and do not want to challenge it. 

There was also work focusing on what could happen before /dev/urandom being properly seeded and its internal state being completely unpredictable and unknown to the attacker. [CITE: Ps and Qs] successfully caught such a vulnerability that on some embedded devices the ssh-keygen was invoked before /dev/urandom was ready, resulting in identical or factorable RSA private keys. Our goal is to identify a similar vulnerability, but on Android devices which are more complicated and not investigated. And our result and conclusion turns out to be different from that for embedded devices: we find that the user applications do not suffer from the low entropy issue while it is the anti-buffer overflow mechanisms, implemented in the kernel and other parts of Android framework that suffer from it.

ALSR was proposed and researched in some work. [CITE: 32-bit ALSR useless] showed that on an 32-bit architecture ALSR could do little help. Our work confirms further this point: the ALSR implemented in Linux for the arm architecture provides only 8 bits of entropy for each mapped memory range and the base address could be leaked by a side channel.


